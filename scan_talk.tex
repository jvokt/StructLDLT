% Created 2013-11-08 Fri 13:29
\documentclass[bigger]{beamer}
\usepackage[utf8]{inputenc}
\usepackage[T1]{fontenc}
\usepackage{fixltx2e}
\usepackage{graphicx}
\usepackage{longtable}
\usepackage{float}
\usepackage{wrapfig}
\usepackage{soul}
\usepackage{textcomp}
\usepackage{marvosym}
\usepackage{wasysym}
\usepackage{latexsym}
\usepackage{amssymb}
\usepackage{hyperref}
\tolerance=1000
\providecommand{\alert}[1]{\textbf{#1}}

\title{Approximating matrices with multiple symmetries: with an application to quantum chemistry}
\author{Joseph Vokt, Senior, Computer Science\\jpv52@cornell.edu\\Joint-work with \\Professor Charles Van Loan, Computer Science\\cv@cs.cornell.edu}
\date{\today}
\hypersetup{
  pdfkeywords={},
  pdfsubject={},
  pdfcreator={Emacs Org-mode version 7.9.3f}}

\usetheme{Copenhagen}\usecolortheme{default}\definecolor{color17}{rgb}{.8,0.0,1}\setbeamertemplate{footline}[page number]{}\setbeamertemplate{navigation symbols}{}\definecolor{dark_green}{rgb}{0.0, 0.6, 0.0}\input{TensorCommands}
\begin{document}

\maketitle



\section{}
\begin{frame}
\frametitle{Themes for this talk}
\label{sec-1-1}
\begin{itemize}

\item Symmetry
\label{sec-1-1-1}%

\item Low-rank approximation
\label{sec-1-1-2}%

\item Preserving symmetry for low-rank approximation
\label{sec-1-1-3}%

\item Bridging the gap from matrix to tensor
\label{sec-1-1-4}%
\end{itemize} % ends low level
\end{frame}
\begin{frame}
\frametitle{ERI Tensor Symmetry: Background}
\label{sec-1-2}
\begin{itemize}

\item The electron repulsion integral (ERI) tensor is defined by
\label{sec-1-2-1}%
\begin{align*}
&{\cal A}(i,j,k,\ell)
=\int d\textbf{x}_1d\textbf{x}_2 \chi_i(\textbf{x}_1)^*\chi_j(\textbf{x}_1)\frac{1}{r_{12}} \chi_k(\textbf{x}_2)^*\chi_{\ell}(\textbf{x}_2)
\end{align*}

\item $\chi(\textbf{x})$ is a molecular orbital that describes an electron's motion at coordinate $\textbf{x}=(r,\theta,\phi,\omega)$ where $\omega$ is the spin coordinate
\label{sec-1-2-2}%

\item This tensor describes two-body Coulomb interaction in molecular electronic structure theory
\label{sec-1-2-3}%

\item Source: Szabo and Ostlund's \emph{Modern Quantum Chemistry}
\label{sec-1-2-4}%
\end{itemize} % ends low level
\end{frame}
\begin{frame}
\frametitle{ERI Tensor Symmetry: Background}
\label{sec-1-3}
\begin{itemize}

\item For real orbitals, we have the following symmetries:
\label{sec-1-3-1}%
\begin{align*}
&{\cal A}(i,j,k,\ell) = {\cal A}(k,\ell,i,j) = {\cal A}(j,i,\ell,k) = {\cal A}(\ell,k,j,i)\\
&={\cal A}(j,i,k,\ell) = {\cal A}(\ell,k,i,j) = {\cal A}(i,j,\ell,k) = {\cal A}(k,\ell,j,i)
\end{align*}

\item For our purposes, we will be primarily concerned with
\label{sec-1-3-2}%
\begin{equation*}
{\cal A}(i,j,k,\ell) = {\cal A}(j,i,k,\ell) = {\cal A}(i,j,\ell,k) = {\cal A}(k,\ell,i,j)
\end{equation*}
\end{itemize} % ends low level
\end{frame}
\begin{frame}
\frametitle{ERI Tensor Symmetry: Problem Statement}
\label{sec-1-4}

Compute $\mu$, the coulomb energy
\begin{align*}
\mu &= \sum_{i=1}^{n}  \sum_{j=1}^{n}  \sum_{k=1}^{n}  \sum_{\ell=1}^{n}
{\cal A}(i,j,k,\ell) v_{i}v_{j}v_{k}v_{\ell}\\
&= (v\kron v)^{T} A (v \kron v)
\end{align*}
where \inv{v}{n} and \inv{\cal A}{n\times n \times n \times n} is the fourth-order ERI tensor and $A$ is the tensor unfolding.
\end{frame}
\begin{frame}
\frametitle{ERI Tensor Symmetry: Motivation}
\label{sec-1-5}
\begin{itemize}

\item Applications of ERI tensors
\label{sec-1-5-1}%
\begin{itemize}

\item \emph{Ab initio} protein folding simulation
\label{sec-1-5-1-1}%

\item Rational therapeutic drug design
\label{sec-1-5-1-2}%

\item Engineering nano-scale devices
\label{sec-1-5-1-3}%
\end{itemize} % ends low level

\item The more electrons, the higher the dimensionality $d$
\label{sec-1-5-2}%
\begin{itemize}

\item $d$-body problem at the heart of quantum chemistry
\label{sec-1-5-2-1}%
\end{itemize} % ends low level
\end{itemize} % ends low level
\end{frame}
\begin{frame}
\frametitle{ERI Tensor Symmetry: Goal}
\label{sec-1-6}
\begin{itemize}

\item Goal: a low-rank approximation which preserves the original structure \emph{without} using the SVD, while minimizing ERI evaluations
\label{sec-1-6-1}%
\begin{itemize}

\item Why low-rank?
\label{sec-1-6-1-1}%
\begin{itemize}

\item Data sparsity
\label{sec-1-6-1-1-1}%
\end{itemize} % ends low level

\item Why preserve structure?
\label{sec-1-6-1-2}%
\begin{itemize}

\item Numerical motivation: good to maintain data sparsity
\label{sec-1-6-1-2-1}%

\item Physical motivation: must not violate the Pauli principle
\label{sec-1-6-1-2-2}%
\end{itemize} % ends low level

\item Why not use the SVD?
\label{sec-1-6-1-3}%
\begin{itemize}

\item The SVD of an $n^2xn^2$ matrix takes $O(n^6)$ flops
\label{sec-1-6-1-3-1}%
\end{itemize} % ends low level

\item Why minimize ERI evaluations?
\label{sec-1-6-1-4}%
\begin{itemize}

\item ERI evaluations are expensive to compute
\label{sec-1-6-1-4-1}%
\end{itemize} % ends low level
\end{itemize} % ends low level
\end{itemize} % ends low level
\end{frame}
\begin{frame}
\frametitle{Centrosymmetry: A tale of two symmetries}
\label{sec-1-7}

A persymmetric matrix is symmetric about its antidiagonal:
\[
A \:=\: 
\left[ 
\begin{array}{cccc}
1 & 2 & 3 & \colorbox{red}{4}\\
5 & 6 & \colorbox{red}{7} & 3 \\
8 & \colorbox{red}{9} & 6 & 2 \\
\colorbox{red}{10} & 8 & 5 & 1
\end{array} 
\right].
\]
\end{frame}
\begin{frame}
\frametitle{Centrosymmetry: A tale of two symmetries}
\label{sec-1-8}

A matrix \inm{A}{n}{n} is persymmetric if  $A^{T} = E_{n}AE_{n}$ where \inm{E_{n}}{n}{n} is
the $n$-by-$n$ exchange permutation, e.g.,
\[
E_{4} \:=\: 
\left[ 
\begin{array}{cccc} 
0 & 0 & 0 & 1\\
0 & 0 & 1 & 0 \\
0 & 1 & 0 & 0 \\
1 & 0 & 0 & 0
\end{array} 
\right].
\]
\end{frame}
\begin{frame}
\frametitle{Centrosymmetry: A tale of two symmetries}
\label{sec-1-9}

A centrosymmetric matrix is a symmetric persymmetric matrix $A = E_{n}AE_{n}$ e.g.,
\[
A \:=\: 
\left[ 
\begin{array}{cccc} 
1 & 2 & \colorbox{green}{3} & 4\\
2 & 5 & 6 & \colorbox{green}{3} \\
\colorbox{green}{3} & 6 & 5 & 2 \\
4 & \colorbox{green}{3} & 2 & 1
\end{array} 
\right]
\]
\end{frame}
\begin{frame}
\frametitle{Centrosymmetry: Block Structure}
\label{sec-1-10}

If \inm{A}{n}{n} is blocked as
\[A \:=\: \left[ \begin{array}{cc} A_{11} & A_{12} \\ A_{21} & A_{22} \end{array} \right]
\qquad \inm{A_{ij}}{m}{m}\]
and $A$ is centrosymmetric, where $n$ is even, then \[A \:= \: \left[ \begin{array}{cc} A_{11} & A_{12} \\ E_{m}A_{12}E_{m} & E_{m}A_{11}E_{m}
\end{array} \right]\]
\end{frame}
\begin{frame}
\frametitle{Centrosymmetry: Block Diagonalization}
\label{sec-1-11}

If we define $Q_E$ as
\[Q_{E} \:=\: \frac{1}{\sqrt{2}} 
\left[ \begin{array}{c|c} I_{m} & I_{m} \\ E_{m} & -E_{m} \end{array} \right]
\:=\: \left[ \begin{array}{c|c} Q_{+} & Q_{-} \end{array} \right]\]
then $Q_{E}$ is orthogonal and
\[Q_{E}^{T} A Q_{E} \:=\: 
\left[ \begin{array}{cc} A_{11} + A_{12}E_{m} & 0 \\ 0 & A_{11}-A_{12}E_{m} \end{array} \right]\]
is the block diagonalization of A.
\end{frame}
\begin{frame}
\frametitle{Centrosymmetry: Block Factorizations}
\label{sec-1-12}
\begin{itemize}

\item When $A$ is positive definite, diagonal blocks are too
\label{sec-1-12-1}%

\item Stable $LDL^{T}$ factorizations:\\
\label{sec-1-12-2}%
\[P_{+}(A_{11}+A_{12}E_{m})P_{+}^{T}  =  L_{+}D_{+}L_{+}^{T}\]\[P_{-}(A_{11}-A_{12}E_{m})P_{-}^{T}  =  L_{-}D_{-}L_{-}^{T}\]


\item Obtaining both factorizations takes $2(\frac{1}{3}(\frac{n}{2})^3)=\frac{1}{4}(\frac{n^3}{3})$ flops
\label{sec-1-12-3}%

\item Centrosymmetric $A$ lets you solve $Ax=b$ in $\frac{1}{4}$ of the work as compared to non-symmetric $A$
\label{sec-1-12-4}%
\end{itemize} % ends low level
\end{frame}
\begin{frame}
\frametitle{Centrosymmetry: Low-rank approximation}
\label{sec-1-13}

Let
\[
Y_{+} \:=\: Q_{+}P_{+}^{T}L_{+} \:=\: [ \:y^{(1)}_{+} \:| \: \cdots \:|\:y_{+}^{(m_+)} \:]
\]
and
\[
Y_{-} \:=\: Q_{-}P_{-}^{T}L_{-} \:=\: [ \:y^{(1)}_{-} \:| \: \cdots \:|\:y_{-}^{(m_-)} \:]
\]
\end{frame}
\begin{frame}
\frametitle{Centrosymmetry: Low-rank approximation}
\label{sec-1-14}

\begin{eqnarray*}
A&=& Q_{E} \left[ \begin{array}{cc} A_{11} + A_{12}E_{m} & 0 \\ 0 & A_{11}-A_{12}E_{m} \end{array} \right] Q_{E}^{T}\\
&=&  Q_{+}(A_{11}+A_{12}E_{m}) Q_{+}^{T} \:+\: Q_{-}(A_{11}-A_{12}E_{m}) Q_{-}^{T}\\
&=& Y_{+}D_{+} Y_{+}^{T} \:+\: Y_{-}D_{-} Y_{-}^{T}\\
&=& \sum_{k=1}^{m_{+}} d_{+}^{(k)} y_{+}^{(k)}[y_{+}^{(k)}]^{T} \;+\;
    \sum_{k=1}^{m_{-}} d_{-}^{(k)} y_{-}^{(k)}[y_{-}^{(k)}]^{T}
\end{eqnarray*}
\begin{itemize}

\item By terminating these sums early, we obtain low rank approximations that are also centrosymmetric.
\label{sec-1-14-1}%
\end{itemize} % ends low level
\end{frame}
\begin{frame}
\frametitle{ERI Tensor Symmetry: Block Structure}
\label{sec-1-15}
\begin{itemize}

\item Unfold ${\cal A}(i,j,k,\ell)$ as $[A_{k,\ell}]_{i,j}$
\label{sec-1-15-1}%
\begin{equation*}
A=\mathcal{A}_{[1, 3]\times[2, 4]}
\end{equation*}

\item ${\cal A}(i,j,k,\ell)$ is the $(i,j)$ entry of the $(k,\ell)$ block of $A$
\label{sec-1-15-2}%

\item Assume $n=3$ for the following examples
\label{sec-1-15-3}%
\end{itemize} % ends low level
\end{frame}
\begin{frame}
\frametitle{ERI Tensor Symmetry: Block Structure}
\label{sec-1-16}
\begin{itemize}

\item If ${\cal A}(i,j,k,\ell) = {\cal A}(j,i,k,\ell)$, then $[A_{k,\ell}]_{i,j}=[A_{k,\ell}]_{j,i}$
\label{sec-1-16-1}%

\item Symmetric blocks: each 3x3 block matrix is symmetric along its diagonal\\
\label{sec-1-16-2}%
\[
\begin{array}{ccc|ccc|ccc}
1 &  2 &  3 &  19 &  20 &  21 &  37 &  38 &  39 \\
2 &  4 &  5 &  20 &  22 &  23 &  38 &  40 &  41 \\
3 &  5 &  6 &  21 &  23 &  24 &  39 &  41 &  42 \\
\hline
7 &  8 &  9 &  \colorbox{red}{25} &  \colorbox{yellow}{26} &  \colorbox{cyan}{27} &  43 &  44 &  45 \\
8 &  10 &  11 &  \colorbox{yellow}{26} &  \colorbox{orange}{28} &  \colorbox{green}{29} &  44 &  46 &  47 \\
9 &  11 &  12 &  \colorbox{cyan}{27} &  \colorbox{green}{29} &  \colorbox{color17}{30} &  45 &  47 &  48 \\
\hline
13 &  14 &  15 &  31 &  32 &  33 &  49 &  50 &  51 \\
14 &  16 &  17 &  32 &  34 &  35 &  50 &  52 &  53 \\
15 &  17 &  18 &  33 &  35 &  36 &  51 &  53 &  54 \\
\end{array}
\]
\end{itemize} % ends low level
\end{frame}
\begin{frame}
\frametitle{ERI Tensor Symmetry: Block Structure}
\label{sec-1-17}
\begin{itemize}

\item If ${\cal A}(i,j,k,\ell) = {\cal A}(i,j,\ell,k)$, then $[A_{k,\ell}]_{i,j}=[A_{\ell,k}]_{i,j}$
\label{sec-1-17-1}%

\item Block symmetry: the 9x9 matrix is symmetric at the 3x3 block matrix level\\
\label{sec-1-17-2}%
\vspace{1em}
\begin{array}{ccc|ccc|ccc}
\colorbox{red}{1} &  \colorbox{red}{4} &  \colorbox{red}{7} &  \colorbox{yellow}{10} &  \colorbox{yellow}{13} &  \colorbox{yellow}{16} &  \colorbox{cyan}{19} &  \colorbox{cyan}{22} &  \colorbox{cyan}{25} \\
\colorbox{red}{2} &  \colorbox{red}{5} &  \colorbox{red}{8} &  \colorbox{yellow}{11} &  \colorbox{yellow}{14} &  \colorbox{yellow}{17} &  \colorbox{cyan}{20} &  \colorbox{cyan}{23} &  \colorbox{cyan}{26} \\
\colorbox{red}{3} &  \colorbox{red}{6} &  \colorbox{red}{9} &  \colorbox{yellow}{12} &  \colorbox{yellow}{15} &  \colorbox{yellow}{18} &  \colorbox{cyan}{21} &  \colorbox{cyan}{24} &  \colorbox{cyan}{27} \\
\colorbox{yellow}{10} &  \colorbox{yellow}{13} &  \colorbox{yellow}{16} &  \colorbox{orange}{28} &  \colorbox{orange}{31} &  \colorbox{orange}{34} &  \colorbox{green}{37} &  \colorbox{green}{40} &  \colorbox{green}{43} \\
\colorbox{yellow}{11} &  \colorbox{yellow}{14} &  \colorbox{yellow}{17} &  \colorbox{orange}{29} &  \colorbox{orange}{32} &  \colorbox{orange}{35} &  \colorbox{green}{38} &  \colorbox{green}{41} &  \colorbox{green}{44} \\
\colorbox{yellow}{12} &  \colorbox{yellow}{15} &  \colorbox{yellow}{18} &  \colorbox{orange}{30} &  \colorbox{orange}{33} &  \colorbox{orange}{36} &  \colorbox{green}{39} &  \colorbox{green}{42} &  \colorbox{green}{45} \\
\colorbox{cyan}{19} &  \colorbox{cyan}{22} &  \colorbox{cyan}{25} &  \colorbox{green}{37} &  \colorbox{green}{40} &  \colorbox{green}{43} &  \colorbox{color17}{46} &  \colorbox{color17}{49} &  \colorbox{color17}{52} \\
\colorbox{cyan}{20} &  \colorbox{cyan}{23} &  \colorbox{cyan}{26} &  \colorbox{green}{38} &  \colorbox{green}{41} &  \colorbox{green}{44} &  \colorbox{color17}{47} &  \colorbox{color17}{50} &  \colorbox{color17}{53} \\
\colorbox{cyan}{21} &  \colorbox{cyan}{24} &  \colorbox{cyan}{27} &  \colorbox{green}{39} &  \colorbox{green}{42} &  \colorbox{green}{45} &  \colorbox{color17}{48} &  \colorbox{color17}{51} &  \colorbox{color17}{54} \\
\end{array}
\end{itemize} % ends low level
\end{frame}
\begin{frame}
\frametitle{ERI Tensor Symmetry: Block Structure}
\label{sec-1-18}
\begin{itemize}

\item If ${\cal A}(i,j,k,\ell) = {\cal A}(k,\ell,i,j)$, then $[A_{k,\ell}]_{i,j}=[A_{i,j}]_{k,\ell}$
\label{sec-1-18-1}%

\item Perfect shuffle permutation symmetry: entry $(i,j)$ in the $(k,\ell)$ block is entry $(k,\ell)$ in the $(i,j)$ block\\\\
\label{sec-1-18-2}%
\label{sec-3-4-1}%
\vspace{1em}
\hspace{3em}
\begin{array}{ccc|ccc|ccc}
\colorbox{violet}{1} &  \colorbox{orange}{4} &  \colorbox{cyan}{7} &  \colorbox{orange}{4} &  25 &  28 &  \colorbox{cyan}{7} &  28 &  40 \\
\colorbox{yellow}{2} &  \colorbox{green}{5} &  \colorbox{color17}{8} &  12 &  26 &  29 &  15 &  33 &  41 \\
\colorbox{red}{3} &  \colorbox{brown}{6} &  \colorbox{black}{\textcolor{white}{9}} &  19 &  27 &  30 &  22 &  37 &  42 \\
\hline
\colorbox{yellow}{2} &  12 &  15 &  \colorbox{green}{5} &  26 &  33 &  \colorbox{color17}{8} &  29 &  41 \\
10 &  13 &  16 &  13 &  31 &  34 &  16 &  34 &  43 \\
11 &  14 &  17 &  20 &  32 &  35 &  23 &  38 &  44 \\
\hline
\colorbox{red}{3} &  19 &  22 &  \colorbox{brown}{6} &  27 &  37 &  \colorbox{black}{\textcolor{white}{9}} &  30 &  42 \\
11 &  20 &  23 &  14 &  32 &  38 &  17 &  35 &  44 \\
18 &  21 &  24 &  21 &  36 &  39 &  24 &  39 &  45 \\
\end{array}

\end{itemize} % ends low level
\end{frame}
\begin{frame}
\frametitle{ERI Tensor Symmetry: Block Structure}
\label{sec-1-19}
\begin{itemize}

\item With all eight symmetries\\
\label{sec-1-19-1}%
\vspace{1em}
\begin{array}{ccc|ccc|ccc}
1 &  \colorbox{red}{2} &  3 &  \colorbox{red}{2} &  \colorbox{yellow}{7} &  \colorbox{cyan}{8} &  3 &  \colorbox{cyan}{8} &  12 \\
\colorbox{red}{2} &  4 &  5 &  \colorbox{yellow}{7} &  \colorbox{orange}{9} &  \colorbox{green}{10} &  \colorbox{cyan}{8} &  13 &  14 \\
3 &  5 &  6 &  \colorbox{cyan}{8} &  \colorbox{green}{10} &  \colorbox{color17}{11} &  12 &  14 &  15 \\
\hline
\colorbox{red}{2} &  \colorbox{yellow}{7} &  \colorbox{cyan}{8} &  4 &  \colorbox{orange}{9} &  13 &  5 &  \colorbox{green}{10} &  14 \\
\colorbox{yellow}{7} &  \colorbox{orange}{9} &  \colorbox{green}{10} &  \colorbox{orange}{9} &  16 &  17 &  \colorbox{green}{10} &  17 &  19 \\
\colorbox{cyan}{8} &  \colorbox{green}{10} &  \colorbox{color17}{11} &  13 &  17 &  18 &  14 &  19 &  20 \\
\hline
3 &  \colorbox{cyan}{8} &  12 &  5 &  \colorbox{green}{10} &  14 &  6 &  \colorbox{color17}{11} &  15 \\
\colorbox{cyan}{8} &  13 &  14 &  \colorbox{green}{10} &  17 &  19 &  \colorbox{color17}{11} &  18 &  20 \\
12 &  14 &  15 &  14 &  19 &  20 &  15 &  20 &  21 \\
\end{array}

\end{itemize} % ends low level
\end{frame}
\begin{frame}
\frametitle{ERI Tensor Symmetry: Block Diagonalization}
\label{sec-1-20}
\begin{itemize}

\item Define $Q=\left[Q_{\mbox{\tiny sym}}|Q_{\mbox{\tiny skew}} \right]$ such that
\label{sec-1-20-1}%
\begin{itemize}

\item $Q_{\mbox{\tiny sym}}$ and $Q_{\mbox{\tiny skew}}$ are sparse orthonormal bases for vectorized symmetric and skew-symmetric matrices respectively
\label{sec-1-20-1-1}%
\end{itemize} % ends low level

\item Then $Q$ is orthogonal and\\
\label{sec-1-20-2}%
\[
Q^{T}AQ \:=\: 
\left[ \begin{array}{cc} A_{\mbox{\tiny sym}} & 0 \\ 0 & A_{\mbox{\tiny skew}} \rule{0pt}{13pt}
\end{array}\right]
\]
is the block diagonalization of A.
\end{itemize} % ends low level
\end{frame}
\begin{frame}
\frametitle{ERI Tensor Symmetry: Block Diagonalization}
\label{sec-1-21}

When $n=3$, \inm{Q}{9}{9}
\hspace{-2em}
\begin{equation*}
Q_{9} = \frac{1}{\sqrt{2}}
\left[
\begin{array}{cccccc|rrr}
\sqrt{2} & 0 & 0  & \textcolor{red}{0} & 0 &0 &\textcolor{blue}{0} &0 &0 \\
0 & 0 & 0  & \textcolor{red}{1} & 0 &0 &\textcolor{blue}{1} &0 &0 \\
0 & 0 & 0  & \textcolor{red}{0} & 1 &0 &\textcolor{blue}{0} &1 &0 \\
0 & 0 & 0  & \textcolor{red}{1} & 0 &0 &\textcolor{blue}{-1} &0 &0 \\
0 & \sqrt{2} & 0  & \textcolor{red}{0} & 0 &0 &\textcolor{blue}{0} &0 &0 \\
0 & 0 & 0  & \textcolor{red}{0} & 0 &1 &\textcolor{blue}{0} & 0 &1 \\
0 & 0 & 0  & \textcolor{red}{0} & 1 &0 &\textcolor{blue}{0} &0 &0 \\
0 & 0 & 0  & \textcolor{red}{0} & 0 &1 &\textcolor{blue}{0} &{-1} &{-1} \\
0 & 0 & \sqrt{2}  & \textcolor{red}{0} & 0 &0 &\textcolor{blue}{0} &0 &0 
\end{array}
\right]
= \left[ Q_{\mbox{\tiny sym}} | Q_{\mbox{\tiny skew}} \right]
\end{equation*}
\hspace{-3em}
\[
Q_{9}(:,4) \:\equiv \: \left[ \begin{array}{ccc} \textcolor{red}{0} & \textcolor{red}{1} & \textcolor{red}{0} \\ \textcolor{red}{1} & \textcolor{red}{0} & \textcolor{red}{0} \\ \textcolor{red}{0} & \textcolor{red}{0} & \textcolor{red}{0} \end{array} \right]
\qquad
Q_{9}(:,7) \:\equiv \: \left[ \begin{array}{rrr} \textcolor{blue}{0} & \textcolor{blue}{-1} & \textcolor{blue}{0} \\ \textcolor{blue}{1} & \textcolor{blue}{0} & \textcolor{blue}{0} \\ \textcolor{blue}{0} & \textcolor{blue}{0} & \textcolor{blue}{0} \end{array} \right]
\]
\end{frame}
\begin{frame}
\frametitle{ERI Tensor Symmetry: Block Factorizations}
\label{sec-1-22}
\begin{itemize}

\item When $A$ is positive semi-definite, $A_{\mbox{\tiny sym}}$ and $A_{\mbox{\tiny skew}}$ are too
\label{sec-1-22-1}%

\item Stable symmetric-pivoting $LDL^{T}$ factorizations:
\label{sec-1-22-2}%
\begin{align*}
P_{\mbox{\tiny sym}}A_{\mbox{\tiny sym}}P_{\mbox{\tiny sym}}^{T}  &=  L_{\mbox{\tiny sym}}D_{\mbox{\tiny sym}}L_{\mbox{\tiny sym}}^{T}\\
P_{\mbox{\tiny skew}}A_{\mbox{\tiny skew}}P_{\mbox{\tiny skew}}^{T}  &=  L_{\mbox{\tiny skew}}D_{\mbox{\tiny skew}}L_{\mbox{\tiny skew}}^{T}
\end{align*}

\item Compute a lazy-evaluation, symmetric-pivoting, rank-revealing $LDL^T$ factorization on each block
\label{sec-1-22-3}%

\item Without rank-revealing $LDL^T$
\label{sec-1-22-4}%
\begin{itemize}

\item $n^6/24$ flops for one factorization
\label{sec-1-22-4-1}%

\item $n^6/12$ flops total
\label{sec-1-22-4-2}%
\end{itemize} % ends low level

\item With rank-revealing $LDL^T$, where $r$ is the rank of $A$
\label{sec-1-22-5}%
\begin{itemize}

\item $r^2(n(n+1)/2)=r^2(n^2/2+n/2)$ for one factorization
\label{sec-1-22-5-1}%

\item $r^2n^2$ flops total
\label{sec-1-22-5-2}%
\end{itemize} % ends low level
\end{itemize} % ends low level
\end{frame}
\begin{frame}
\frametitle{ERI Tensor Symmetry: Low-rank approximation}
\label{sec-1-23}

Let
\[
Y_{\mbox{\tiny sym}} \:=\: Q_{\mbox{\tiny sym}}P_{\mbox{\tiny sym}}^{T}L_{\mbox{\tiny sym}} \:=\: [ \:y^{(1)}_{\mbox{\tiny sym}} \:| \: \cdots \:|\:y_{\mbox{\tiny sym}}^{(m_{\mbox{\tiny sym}})} \:]
\]
and
\[
Y_{\mbox{\tiny skew}} \:=\: Q_{\mbox{\tiny skew}}P_{\mbox{\tiny skew}}^{T}L_{\mbox{\tiny skew}} \:=\: [ \:y^{(1)}_{\mbox{\tiny skew}} \:| \: \cdots \:|\:y_{\mbox{\tiny skew}}^{(m_{\mbox{\tiny skew}})} \:]
\]
\end{frame}
\begin{frame}
\frametitle{ERI Tensor Symmetry: Low-rank approximation}
\label{sec-1-24}

\begin{align*}
A&= Q \left[ \begin{array}{cc} A_{\mbox{\tiny sym}} & 0 \\ 0 & A_{\mbox{\tiny skew}} \end{array} \right] Q^{T}\\
&=  Q_{\mbox{\tiny sym}}A_{\mbox{\tiny sym}} Q_{\mbox{\tiny sym}}^{T} \:+\: Q_{\mbox{\tiny skew}}A_{\mbox{\tiny skew}} Q_{\mbox{\tiny skew}}^{T}\\
&= Y_{\mbox{\tiny sym}}D_{\mbox{\tiny sym}} Y_{\mbox{\tiny sym}}^{T} \:+\: Y_{\mbox{\tiny skew}}D_{\mbox{\tiny skew}} Y_{\mbox{\tiny skew}}^{T}\\
&= \sum_{k=1}^{m_{\mbox{\tiny sym}}} d_{\mbox{\tiny sym}}^{(k)} y_{\mbox{\tiny sym}}^{(k)}[y_{\mbox{\tiny sym}}^{(k)}]^{T} \;+\;
    \sum_{k=1}^{m_{\mbox{\tiny skew}}} d_{\mbox{\tiny skew}}^{(k)} y_{\mbox{\tiny skew}}^{(k)}[y_{\mbox{\tiny skew}}^{(k)}]^{T}\\
&= \sum_{k=1}^{m_{\mbox{\tiny sym}}} d_{\mbox{\tiny sym}}^{(k)} B_k \kron B_k+\sum_{k=1}^{m_{\mbox{\tiny skew}}} d_{\mbox{\tiny skew}}^{(k)} C_k \kron C_k
\end{align*}
\begin{itemize}

\item $B_k,C_k$ is the $n\times n$ reshaping of $y_{\mbox{\tiny sym}}^{(k)},y_{\mbox{\tiny skew}}^{(k)}$ respectively
\label{sec-1-24-1}%

\item By terminating these sum early, we obtain low-rank approximations with ERI tensor symmetry.
\label{sec-1-24-2}%

\end{itemize} % ends low level
\end{frame}
\begin{frame}
\frametitle{ERI Tensor Symmetry: Energy calculation}
\label{sec-1-25}

\begin{align*}
\mu 
&= \sum_{i=1}^n \sum_{j=1}^n \sum_{k=1}^n \sum_{l=1}^n \mathcal{A}(i,j,k,l)v_iv_jv_kv_l\\
&= (v \otimes v)^T A (v \otimes v)\\
&= (v\kron v)^{T}(\sum_{k=1}^{m_{\mbox{\tiny sym}}} d_{\mbox{\tiny sym}}^{(k)} B_k \kron B_k+\sum_{k=1}^{m_{\mbox{\tiny skew}}} d_{\mbox{\tiny skew}}^{(k)} C_k \kron C_k)(v \kron v)\\
&= \sum_{k=1}^{m_{\mbox{\tiny sym}}} d_{\mbox{\tiny sym}}^{(k)} \cdot (v \otimes v)^T(B_k\otimes B_k)(v \otimes v)^T \\ 
&+ \sum_{k=1}^{m_{\mbox{\tiny skew}}} d_{\mbox{\tiny sym}}^{(k)} \cdot (v \otimes v)^T(B_k\otimes B_k)(v \otimes v)^T\\
&= \sum_{k=1}^{m_{\mbox{\tiny sym}}} d_{\mbox{\tiny sym}}^{(k)} \cdot (v^TB_kv)^2+ \sum_{k=1}^{m_{\mbox{\tiny skew}}} d_{\mbox{\tiny skew}}^{(k)} \cdot (v^TC_kv)^2
\end{align*}
\end{frame}
\begin{frame}
\frametitle{ERI Tensor Symmetry: Energy calculation}
\label{sec-1-26}
\begin{itemize}

\item $A_{\mbox{\tiny skew}}$ factorization isn't necessary
\label{sec-1-26-1}%

\item Let $C_k$ be the $n\times n$ reshaping of $\inv{y_{\mbox{\tiny skew}}^{(k)}}{n^2}$
\label{sec-1-26-2}%

\item By construction $C_k=-C_k^T$ is skew-symmetric, so $v^TC_kv=v^TC_k^Tv=-v^TC_kv=0$
\label{sec-1-26-3}%

\item The second summation is zero
\label{sec-1-26-4}%
\end{itemize} % ends low level
\end{frame}
\begin{frame}
\frametitle{MATLAB Implementation}
\label{sec-1-27}
\begin{itemize}

\item EnergyCalculation
\label{sec-1-27-1}%
\begin{itemize}

\item Calls StructLDLT on the tensor unfolding
\label{sec-1-27-1-1}%

\item Does the energy calculation
\label{sec-1-27-1-2}%
\end{itemize} % ends low level

\item StructLDLT
\label{sec-1-27-2}%
\begin{itemize}

\item Computes diagonal block $A_{\mbox{\tiny sym}}$ (and $A_{\mbox{\tiny skew}}$ optionally)
\label{sec-1-27-2-1}%
\begin{itemize}

\item Uses QsymT (next slide)
\label{sec-1-27-2-1-1}%
\end{itemize} % ends low level

\item Calls LazyLDLT on $A_{\mbox{\tiny sym}}$ (and on $A_{\mbox{\tiny skew}}$ optionally)
\label{sec-1-27-2-2}%

\item Returns $Y_{\mbox{\tiny sym}}$ (and $Y_{\mbox{\tiny skew}}$ optionally)
\label{sec-1-27-2-3}%
\begin{itemize}

\item Uses Qsym
\label{sec-1-27-2-3-1}%
\end{itemize} % ends low level
\end{itemize} % ends low level

\item LazyLDLT
\label{sec-1-27-3}%
\begin{itemize}

\item Evaluate the diagonal of $A$ [$n^2$ ERIs]
\label{sec-1-27-3-1}%

\item While d(j) is greater than the zero threshold [$r$ loops]
\label{sec-1-27-3-2}%
\begin{itemize}

\item Search d(j:n) for largest diagonal element
\label{sec-1-27-3-2-1}%

\item Swap d(k) and d(j)
\label{sec-1-27-3-2-2}%

\item Update pivot vector, permute rows of L and A
\label{sec-1-27-3-2-3}%

\item Evaluate the next subcolumn of A [$O(n^2)$ ERIs]
\label{sec-1-27-3-2-4}%

\item Compute d(j) and column j of L
\label{sec-1-27-3-2-5}%
\end{itemize} % ends low level
\end{itemize} % ends low level
\end{itemize} % ends low level
\end{frame}
\begin{frame}
\frametitle{MATLAB Implementation: QsymT}
\label{sec-1-28}
\begin{itemize}

\item Goal: find a way to compute $Q_{{\mbox{\tiny sym}}}^Tv$, \inv{v}{n^2} by utilizing the structure of $Q_{{\mbox{\tiny sym}}}^T$
\label{sec-1-28-1}%
\begin{itemize}

\item Sparse matrix-multiply?
\label{sec-1-28-1-1}%
\end{itemize} % ends low level
\end{itemize} % ends low level
\end{frame}
\begin{frame}
\frametitle{MATLAB Implementation: QsymT}
\label{sec-1-29}
\begin{itemize}

\item No matrix-multiplication necessary
\label{sec-1-29-1}%
\begin{align*}
Q_{sym}^T\left[\begin{array}{c}
v_1\\
v_2\\
v_3\\
v_4\\
v_5\\
v_6\\
v_7\\
v_8\\
v_9 \end{array} \right] = Q_{sym}^T vec\left[\begin{array}{c c c}
v_1 & v_4 & v_7\\
v_2 & v_5 & v_8\\
v_3 & v_6 & v_9\end{array}\right] = \left[\begin{array}{c}
v_1 \\
v_5 \\
v_9 \\
(v_2 + v_4)/\sqrt{2}\\
(v_3 + v_7)/\sqrt{2}\\
(v_6 + v_9)/\sqrt{2}
\end{array}\right]
\end{align*}
\end{itemize} % ends low level
\end{frame}
\begin{frame}
\frametitle{MATLAB Implementation: Qsym}
\label{sec-1-30}

\begin{align*}
Q_{sym}\left[\begin{array}{c}
v_1\\
v_2\\
v_3\\
v_4\\
v_5\\
v_6 \end{array} \right] = \left[\begin{array}{c}
v_1\\
v_4\\
v_5\\
v_4\\
v_2\\
v_6\\
v_5\\
v_6\\
v_3 \end{array} \right] = vec\left[\begin{array}{c c c}
v_1 & v_4 & v_5\\
v_4 & v_2 & v_6\\
v_5 & v_6 & v_3\end{array}\right]
\end{align*}
\end{frame}
\begin{frame}
\frametitle{Summary}
\label{sec-1-31}
\begin{itemize}

\item Centrosymmetry and ERI tensor symmetry
\label{sec-1-31-1}%
\begin{itemize}

\item Block structure
\label{sec-1-31-1-1}%

\item Block diagonalization
\label{sec-1-31-1-2}%

\item Block factorizations
\label{sec-1-31-1-3}%

\item Low-rank approximation with original symmetry
\label{sec-1-31-1-4}%
\end{itemize} % ends low level

\item Structure leads to reduced work
\label{sec-1-31-2}%
\begin{itemize}

\item $O(r^2n^2)$ algorithm for computing ERI energy
\label{sec-1-31-2-1}%
\end{itemize} % ends low level
\end{itemize} % ends low level
\end{frame}
\begin{frame}
\frametitle{Future work}
\label{sec-1-32}
\begin{itemize}

\item Parallel implementation
\label{sec-1-32-1}%
\begin{itemize}

\item Exploit block structure
\label{sec-1-32-1-1}%
\end{itemize} % ends low level

\item Sixth order tensor approximation
\label{sec-1-32-2}%
\begin{itemize}

\item First three indices are super-symmetric, last three indices are super-symmetric
\label{sec-1-32-2-1}%

\item Key problem is finding a sparse orthogonal basis to project onto
\label{sec-1-32-2-2}%

\item Group theory approach looks promising
\label{sec-1-32-2-3}%
\end{itemize} % ends low level

\item d-order tensor approximation
\label{sec-1-32-3}%

\item Thanks to Charles Van Loan, the McNair Scholars program, and David Bindel
\label{sec-1-32-4}%
\end{itemize} % ends low level
\end{frame}

\end{document}
